\chapter{Numbers}\label{Numbers}

\section{Natural Numbers $\mathbb{N}$}\label{Natural Numbers}
The set of natural numbers represents the process of counting. Whether or not 0 is part of $\mathbb{N}$ depends on the definition and may vary. If 0 is not included, the set is defined as:

\spacedequation{\mathbb{N} = \{1, 2, 3, 4, \dots, n, n+1, \dots \}.}

How can we perform arithmetic with natural numbers? Addition and multiplication are unrestricted. We say that $\mathbb{N}$ is closed under addition and multiplication. Other operations, such as subtraction and division, are not universally applicable because negative numbers are not part of the natural numbers. A subset of $\mathbb{N}$ is the set of \textbf{prime numbers}, defined as:

\spacedequation{\mathbb{P} = \{1, 2, 3, 5, 7, 11, 13, 17, 19, 23, 29, \dots \}.}

Prime numbers are only divisible by 1 and themselves!

\section{Integers $\mathbb{Z}$}\label{Integers}
The set of integers is obtained by extending the natural numbers to include negative numbers:

\spacedequation{\mathbb{Z} = \{\dots, -3, -2, -1, 0, 1, 2, 3, \dots \}.}

Now subtraction is also possible without restriction.

\section{Rational and Irrational Numbers $\mathbb{Q}$, $\mathbb{R \backslash \mathbb{Q}}$}\label{Rational and Irrational Numbers}
To perform unrestricted division, we need fractions:
\begin{itemize}
    \item \spacedequation{\mathbb{Q}_+ = \left\{ \frac{a}{b} \mid a, b \in \mathbb{N}, b \neq 0 \right\}}
\end{itemize}
Including negative fractions, we get the set of rational numbers:
\begin{itemize}
    \item \spacedequation{\mathbb{Q} = \left\{ \frac{a}{b} \mid a, b \in \mathbb{Z}, b \neq 0 \right\}}
    \item In $\mathbb{Q}$, all basic arithmetic operations are allowed.
    \item $\mathbb{Q}$ includes all positive and negative fractions, as well as all terminating decimal fractions (e.g., -3.75) and repeating decimal fractions (e.g., 0.6666...).
\end{itemize}
One operation is not fully allowed within the rational numbers: taking square roots, since it can lead to infinite numbers that cannot be expressed as fractions. These numbers are called \textbf{irrational numbers}, e.g., 

\spacedequation{\sqrt{2} = 1.41421356\dots}

\section{Real Numbers $\mathbb{R}$}\label{Real Numbers}
By combining the rational and irrational numbers, we get the real numbers $\mathbb{R}$. However, taking square roots of negative numbers is not defined. For example,

\spacedequation{\sqrt{-4}}

is not defined, and such numbers are not included in $\mathbb{R}$.

\section{Complex Numbers $\mathbb{C}$}\label{Complex Numbers}
A complex number \( z \) is represented as a pair of real numbers:

\spacedequation{x + iy \mid x, y \in \mathbb{R}, \quad i = \sqrt{-1}.}

The important feature of the imaginary unit \( i \) is that it allows us to take the square root of negative numbers. A complex number \( z \in \mathbb{C} \) can also be written as the pair \( (x, y) \), where \( x \) is the real part and \( y \) is the imaginary part. Thus, the set of complex numbers $\mathbb{C}$ can be geometrically represented as pairs of real numbers \( (x, y) \) on the complex plane (also called the Gaussian plane), as shown in the figure below.

\begin{figure}[ht]
    \centering
    \includegraphics[width=0.5\textwidth]{../images/Gaußsche_Zahlenebene.png}
    \caption{Gaussian Plane; $c \in \mathbb{C}$ as a real number pair $(x, y)$}
    \label{fig:zahlplane}
\end{figure}

The addition of two complex numbers is defined as:

\spacedequation{z_1 + z_2 = (x_1 + x_2) + (y_1 + y_2) \cdot i}

and the subtraction as:

\spacedequation{z_1 - z_2 = (x_1 - x_2) + (y_1 - y_2) \cdot i.}

Multiplication is defined as:

\spacedequation{z_1 \cdot z_2 = (x_1 + x_2) \cdot (y_1 - y_2) \cdot i = x_1 y_1 + x_1 y_2 \cdot i + y_1 x_2 \cdot i + x_2 y_2 \cdot i^2,}

which simplifies to:

\spacedequation{z_1 \cdot z_2 = (x_1 y_1 - x_2 y_2) + (x_1 y_2 + y_1 x_2) \cdot i.}

If \( z \) is a complex number, then \( z^* \) is its complex conjugate. In the representation below, the real part of \( z \) is reflected. In particular, we have 

\spacedequation{i^* = -i, \quad z^* = x - y \cdot i.}

Multiplying by the complex conjugate gives the magnitude of \( z \):

\spacedequation{|z| = \sqrt{x^2 + y^2}.}

The division of complex numbers is defined as:

\spacedequation{\frac{z_1}{z_2} = \frac{z_1}{z_2} \cdot \frac{z_2^*}{z_2^*}.}

This operation is a bit cumbersome and can be avoided when possible. If it cannot be avoided, multiply the numerator and denominator separately and simplify using the definition of \( i \).

A different representation is possible using polar coordinates:

\spacedequation{z = a + i \cdot b \quad \left. a, b, r \in \mathbb{R}, \theta \in [0, 2\pi] \right| \Leftrightarrow z = r \cdot (\cos(\theta) + i \cdot \sin(\theta)) \Leftrightarrow r \cdot e^{i\theta}.}

\subsection{Special Numbers}\label{Special Numbers}

\subsubsection{$\pi$ - The Circle Constant}\label{Circle Constant}
3.1415926535\dots is the irrational number $\pi$\footnote{Pi's digits omitted for brevity.}. Pi describes the ratio of the circumference to the diameter of a circle. Many formulas involve $\pi$:

\spacedequation{\text{Circumference} \quad U = \pi \cdot d = 2 \cdot \pi \cdot r,}

\spacedequation{\text{Area} \quad A = \pi \cdot r^2,}

\spacedequation{\text{Volume} \quad V = \frac{4}{3} \cdot \pi \cdot r^3.}

\subsubsection{$e$ - Euler's Number}\label{Euler's Number}
Euler's number is the base of the natural logarithm \(\ln(e) = 1\). Euler's number can be approximated as 

\spacedequation{e = 2.71828,}

but like $\pi$, it does not have an exact solution. Named after the Swiss mathematician and physicist Leonhard Euler (1707-1783), \( e \) is crucial for exponential functions.
